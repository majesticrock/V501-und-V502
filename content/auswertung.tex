\section{Auswertung}
\label{sec:Auswertung}

\subsection{Elek.}

\begin{table}
    \centering
    \caption{Gemessene Ablenkungen bei verschiedenen Ablenk- und Beschleunigungsspannungen.}
    \label{tab:elek-ges}
    \begin{subtable}{0.3\textwidth}
\centering
\caption{$U_\text{B} = 250$ V}
\label{tab:elek0}
\begin{tabular}{S[table-format=3.1] S[table-format=2.0]}
\toprule
{$U_d$ / V} & {$D$ / mm} \\
\midrule
-11.1 & 6 \\
 -6.2 & 12 \\
 -1.4 & 18 \\
  3.4 & 24 \\
  8.1 & 30 \\
 12.3 & 36 \\
 17.6 & 42 \\
 21.7 & 48 \\
 25.8 & 54 \\
\bottomrule
\end{tabular}
\end{subtable}
    \begin{subtable}{0.3\textwidth}
\centering
\caption{$U_\text{B} = 300$ V}
\label{tab:elek1}
\begin{tabular}{S[table-format=3.1] S[table-format=2.0]}
\toprule
{$U_d$ / V} & {$D$ / mm} \\
\midrule
-14.4 &  6 \\
-8.7 & 12 \\
-2.7 & 18 \\
3.1 & 24 \\
9.2 & 30 \\
15.0 & 36 \\
20.8 & 42 \\
26.4 & 48 \\
30.9 & 54 \\
\bottomrule
\end{tabular}
\end{subtable}
    \begin{subtable}{0.3\textwidth}
\centering
\caption{$U_\text{B} = 350$ V}
\label{tab:elek2}
\begin{tabular}{S[table-format=3.1] S[table-format=2.0]}
\toprule
{$U_d$ / V} & {$D$ / mm} \\
\midrule
-15.8 &  0 \\
 -8.9 &  6 \\
 -1.9 & 12 \\
  4.6 & 18 \\
 11.2 & 24 \\
 17.7 & 30 \\
 24.2 & 36 \\
 30.9 & 42 \\
 36.5 & 48 \\
\bottomrule
\end{tabular}
\end{subtable}

\vspace{1.2em}

    \begin{subtable}{0.3\textwidth}
\centering
\caption{$U_\text{B} = 400$ V}
\label{tab:elek3}
\begin{tabular}{S[table-format=3.1] S[table-format=2.0]}
\toprule
{$U_d$ / V} & {$D$ / mm} \\
\midrule
-17.8 &  6 \\
-10.1 & 12 \\
 -2.2 & 18 \\
  5.8 & 24 \\
 13.2 & 30 \\
 20.5 & 36 \\
 28.4 & 42 \\
 33.8 & 48 \\
\bottomrule
\end{tabular}
\end{subtable}
    \begin{subtable}{0.3\textwidth}
\centering
\caption{$U_\text{B} = 450$ V}
\label{tab:elek4}
\begin{tabular}{S[table-format=3.1] S[table-format=2.0]}
\toprule
{$U_d$ / V} & {$D$ / mm} \\
\midrule
-20.2 &  0 \\
-11.9 &  6 \\
 -2.7 & 12 \\
  5.7 & 18 \\
 14.7 & 24 \\
 22.8 & 30 \\
 31.6 & 36 \\
 & \\
\bottomrule
\end{tabular}
\end{subtable}
\end{table}

\begin{figure}
  \centering
  \begin{subfigure}{0.49\textwidth}
    \centering
    \includegraphics[width=0.98\textwidth]{build/plot_elektrisch_0.pdf}
    \caption{$U_\text{B} = 250$ V}
  \end{subfigure}
  \begin{subfigure}{0.49\textwidth}
    \centering
    \includegraphics[width=0.98\textwidth]{build/plot_elektrisch_1.pdf}
    \caption{$U_\text{B} = 300$ V}
  \end{subfigure}

  \begin{subfigure}{0.49\textwidth}
    \centering
    \includegraphics[width=0.98\textwidth]{build/plot_elektrisch_2.pdf}
    \caption{$U_\text{B} = 350$ V}
  \end{subfigure}
  \begin{subfigure}{0.49\textwidth}
    \centering
    \includegraphics[width=0.98\textwidth]{build/plot_elektrisch_3.pdf}
    \caption{$U_\text{B} = 400$ V}
  \end{subfigure}

  \begin{subfigure}{0.49\textwidth}
    \centering
    \includegraphics[width=0.98\textwidth]{build/plot_elektrisch_4.pdf}
    \caption{$U_\text{B} = 450$ V}
  \end{subfigure}
  \caption{Plots der Ablenkung $D$ gegen die Ablenkspannung $U_\text{d}$ bei verschiedenen Beschleunigungspannungen $U_\text{B}$.}
  \label{fig:plot-elektrisch}
\end{figure}

\begin{figure}
  \centering
  \includegraphics[width=0.95\textwidth]{build/plot_ub.pdf}
  \caption{Plot der Beschleunigungspannung $U_\text{B}$ gegen die zuvor berechneten Parameter.}
  \label{fig:plt-ub}
\end{figure}


\begin{table}[!htp]
\centering
\caption{Frequenz der Sägezahnspannung und Multiplikator $n$, um die Frequenz der Sinusspannung zu erhalten.}
\label{tab:oszi}
\begin{tabular}{S[table-format=3.2] S[table-format=1.1]}
\toprule
{$\nu_\text{Sägezahn}$ / Hz} & {n} \\
\midrule
25.03 & 0.5 \\
50.03 & 1 \\
100.03 & 2 \\
150.03 & 3 \\
\bottomrule
\end{tabular}
\end{table}

\subsection{Magn.}

\begin{table}
    \centering
    \caption{Gemessene Ablenkungen bei verschiedenen Stromstärken und Beschleunigungsspannungen.}
    \label{tab:magn-ges}
    \begin{subtable}{0.47\textwidth}
\centering
\caption{$U_\text{B} = 250$ V}
\label{tab:magn1}
\begin{tabular}{S[table-format=1.2] S[table-format=2.0]}
\toprule
{$I$ / A} & {$D$ / mm} \\
\midrule
 0.0 &  0 \\
 0.3 &  6 \\
0.65 & 12 \\
 1.0 & 18 \\
1.35 & 24 \\
1.65 & 30 \\
1.95 & 36 \\
2.35 & 42 \\
2.65 & 48 \\
\bottomrule
\end{tabular}
\end{subtable}
    \begin{subtable}{0.47\textwidth}
\centering
\caption{$U_\text{B} = 400$ V}
\label{tab:magn2}
\begin{tabular}{S[table-format=1.2] S[table-format=2.0]}
\toprule
{$I$ / A} & {$D$ / mm} \\
\midrule
 0.0 &  0 \\
0.45 &  6 \\
 0.9 & 12 \\
1.35 & 18 \\
1.75 & 24 \\
2.15 & 30 \\
2.65 & 36 \\
3.05 & 42 \\
 & \\
\bottomrule
\end{tabular}
\end{subtable}
\end{table}