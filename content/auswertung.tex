\section{Auswertung}
\label{sec:Auswertung}

\subsection{Ablenkung im E-Feld}

Zunächst werden die Daten aus \autoref{tab:elek-ges} in Diagrammen aufgetragen.

\begin{table}
    \centering
    \caption{Gemessene Ablenkungen bei verschiedenen Ablenk- und Beschleunigungsspannungen.}
    \label{tab:elek-ges}
    \begin{subtable}{0.3\textwidth}
\centering
\caption{$U_\text{B} = 250$ V}
\label{tab:elek0}
\begin{tabular}{S[table-format=3.1] S[table-format=2.0]}
\toprule
{$U_d$ / V} & {$D$ / mm} \\
\midrule
-11.1 & 6 \\
 -6.2 & 12 \\
 -1.4 & 18 \\
  3.4 & 24 \\
  8.1 & 30 \\
 12.3 & 36 \\
 17.6 & 42 \\
 21.7 & 48 \\
 25.8 & 54 \\
\bottomrule
\end{tabular}
\end{subtable}
    \begin{subtable}{0.3\textwidth}
\centering
\caption{$U_\text{B} = 300$ V}
\label{tab:elek1}
\begin{tabular}{S[table-format=3.1] S[table-format=2.0]}
\toprule
{$U_d$ / V} & {$D$ / mm} \\
\midrule
-14.4 &  0 \\
 -8.7 &  6 \\
 -2.7 & 12 \\
  3.1 & 18 \\
  9.2 & 24 \\
 15.0 & 30 \\
 20.8 & 36 \\
 26.4 & 42 \\
 30.9 & 48 \\
\bottomrule
\end{tabular}
\end{subtable}
    \begin{subtable}{0.3\textwidth}
\centering
\caption{$U_\text{B} = 350$ V}
\label{tab:elek2}
\begin{tabular}{S[table-format=3.1] S[table-format=2.0]}
\toprule
{$U_d$ / V} & {$D$ / mm} \\
\midrule
-15.8 &  6 \\
-8.9 & 12 \\
-1.9 & 18 \\
4.6 & 24 \\
11.2 & 30 \\
17.7 & 36 \\
24.2 & 42 \\
30.9 & 48 \\
36.5 & 54 \\
\bottomrule
\end{tabular}
\end{subtable}

\vspace{1.2em}

    \begin{subtable}{0.3\textwidth}
\centering
\caption{$U_\text{B} = 400$ V}
\label{tab:elek3}
\begin{tabular}{S[table-format=3.1] S[table-format=2.0]}
\toprule
{$U_d$ / V} & {$D$ / mm} \\
\midrule
-17.8 &  6 \\
-10.1 & 12 \\
 -2.2 & 18 \\
  5.8 & 24 \\
 13.2 & 30 \\
 20.5 & 36 \\
 28.4 & 42 \\
 33.8 & 48 \\
\bottomrule
\end{tabular}
\end{subtable}
    \begin{subtable}{0.3\textwidth}
\centering
\caption{$U_\text{B} = 450$ V}
\label{tab:elek4}
\begin{tabular}{S[table-format=3.1] S[table-format=2.0]}
\toprule
{$U_d$ / V} & {$D$ / mm} \\
\midrule
-20.2 &  0 \\
-11.9 &  6 \\
 -2.7 & 12 \\
  5.7 & 18 \\
 14.7 & 24 \\
 22.8 & 30 \\
 31.6 & 36 \\
 & \\
\bottomrule
\end{tabular}
\end{subtable}
\end{table}

Durch diese wird je eine Ausgleichsgerade der Form 

\begin{equation}
  D = a_i \cdot U_\text{d} + b_i
\end{equation}

gelegt. Die somit entstehenden Graphen sind in \autoref{fig:plot-elektrisch} zu sehen.

\begin{figure}
  \centering
  \begin{subfigure}{0.49\textwidth}
    \centering
    \includegraphics[width=0.98\textwidth]{build/plot_elektrisch_0.pdf}
    \caption{$U_\text{B} = 250$ V}
  \end{subfigure}
  \begin{subfigure}{0.49\textwidth}
    \centering
    \includegraphics[width=0.98\textwidth]{build/plot_elektrisch_1.pdf}
    \caption{$U_\text{B} = 300$ V}
  \end{subfigure}

  \begin{subfigure}{0.49\textwidth}
    \centering
    \includegraphics[width=0.98\textwidth]{build/plot_elektrisch_2.pdf}
    \caption{$U_\text{B} = 350$ V}
  \end{subfigure}
  \begin{subfigure}{0.49\textwidth}
    \centering
    \includegraphics[width=0.98\textwidth]{build/plot_elektrisch_3.pdf}
    \caption{$U_\text{B} = 400$ V}
  \end{subfigure}

  \begin{subfigure}{0.49\textwidth}
    \centering
    \includegraphics[width=0.98\textwidth]{build/plot_elektrisch_4.pdf}
    \caption{$U_\text{B} = 450$ V}
  \end{subfigure}
  \caption{Plots der Ablenkung $D$ gegen die Ablenkspannung $U_\text{d}$ bei verschiedenen Beschleunigungspannungen $U_\text{B}$.}
  \label{fig:plot-elektrisch}
\end{figure}

Mittels Python 3.7.0 werden die Koeffizienten als

\begin{align*}
  a_1 &= (1,29 \pm 0,01) \, \frac{\symup{mm}}{\symup{V}} \\
  b_1 &= (13,9 \pm 0,2) \, \symup{mm} \\
  a_2 &= (1,04 \pm 0,01) \, \frac{\symup{mm}}{\symup{V}} \\
  b_2 &= (14,8 \pm 0,2) \, \symup{mm} \\
  a_3 &= (0,914 \pm 0,007) \, \frac{\symup{mm}}{\symup{V}} \\
  b_3 &= (14,0 \pm 0,1) \, \symup{mm} \\
  a_4 &= (0,80 \pm 0,01) \, \frac{\symup{mm}}{\symup{V}} \\
  b_4 &= (13,8 \pm 0,2) \, \symup{mm} \\
  a_5 &= (0,694 \pm 0,004) \, \frac{\symup{mm}}{\symup{V}} \\
  b_5 &= (14,04 \pm 0,07) \, \symup{mm}
\end{align*}

bestimmt. Dabei entsprechen die $a_i$ der jeweiligen Empfindlichkeit $\epsilon_i := \frac{D}{U_\text{d}}$
Diese wird in einem weiteren Plot gegen das Inverse der Beschleunigungspannung $U_\text{B}$ aufgetragen.
Durch diese Werte wird ebenfalls eine Ausgleichsgerade der Form

\begin{equation}
  \epsilon_i = \frac{a_6}{U_\text{B}} + b_6
\end{equation}

gelegt. Der entsprechende Graph ist in \autoref{fig:plt-ub} zu sehen.

\begin{figure}
  \centering
  \includegraphics[width=0.95\textwidth]{build/plot_ub.pdf}
  \caption{Plot der Beschleunigungspannung $U_\text{B}$ gegen die zuvor berechneten Parameter.}
  \label{fig:plt-ub}
\end{figure}

Erneut mittels Python 3.7.0 werden die Koeffizienten als

\begin{align*}
  a_6 &= (330 \pm 10) \, \symup{mm} \\
  b_6 &= (-0,03 \pm 0,03) \, \frac{\symup{mm}}{\symup{V}}
\end{align*}

bestimmt. Zu Vergleichszwecken wird die folgende Größe mit den Theoriewerten berechnet

\begin{equation}
  \frac{p L}{2 d} = 357,5 \, \symup{mm}.
\end{equation}

Dabei sind die Geräteabmessungen gegeben durch

\begin{align*}
  p &= 19  \, \symup{mm} \\
  L &= 143 \, \symup{mm} \\
  d &= 3,8 \, \symup{mm}.
\end{align*}

Weiterhin kann aus den Werten in \autoref{tab:oszi} die Frequenz einer Sinusspannung bestimmt werden.

\begin{table}[!htp]
\centering
\caption{Frequenz der Sägezahnspannung und Multiplikator $n$, um die Frequenz der Sinusspannung zu erhalten.}
\label{tab:oszi}
\begin{tabular}{S[table-format=3.2] S[table-format=1.1]}
\toprule
{$\nu_\text{Sägezahn}$ / Hz} & {n} \\
\midrule
25.03 & 0.5 \\
50.03 & 1 \\
100.03 & 2 \\
150.03 & 3 \\
\bottomrule
\end{tabular}
\end{table}

Dazu werden die Frequenzen der Sägezahnspannung $\nu_\text{Sägezahn}$ dividiert durch $n$ gemittelt.

Damit ergibt sich 

\begin{equation}
  \nu_\text{Sinus} = \frac{1}{N} \cdot \sum_{i=1}^{N} \frac{\nu_{i, \text{Sägezahn}} }{n_i} = 50,029 \, \symup{Hz}
\end{equation}

und deren Abweichung zu 

\begin{equation}
  \Delta \nu_\text{Sinus} = \sqrt{ \frac{1}{N(N-1)} \cdot \sum_{i=1}^{N} \bigg( \frac{\nu_{i, \text{Sägezahn}}}{n_i} - \nu_\text{Sinus} \bigg)^2 } = 0,005 \, \symup{Hz}.
\end{equation}


\subsection{Ablenkspannung im B-Feld}

Es werden zunächst die Daten aus \autoref{tab:magn-ges} in einem Diagramm aufgetragen.

\begin{table}
    \centering
    \caption{Gemessene Ablenkungen bei verschiedenem Spulenstrom und Beschleunigungsspannungen.}
    \label{tab:magn-ges}
    \begin{subtable}{0.47\textwidth}
\centering
\caption{$U_\text{B} = 250$ V}
\label{tab:magn1}
\begin{tabular}{S[table-format=1.2] S[table-format=2.0]}
\toprule
{$I$ / A} & {$D$ / mm} \\
\midrule
 0.0 &  0 \\
 0.3 &  6 \\
0.65 & 12 \\
 1.0 & 18 \\
1.35 & 24 \\
1.65 & 30 \\
1.95 & 36 \\
2.35 & 42 \\
2.65 & 48 \\
\bottomrule
\end{tabular}
\end{subtable}
    \begin{subtable}{0.47\textwidth}
\centering
\caption{$U_\text{B} = 400$ V}
\label{tab:magn2}
\begin{tabular}{S[table-format=1.2] S[table-format=2.0]}
\toprule
{$I$ / A} & {$D$ / mm} \\
\midrule
 0.0 &  0 \\
0.45 &  6 \\
 0.9 & 12 \\
1.35 & 18 \\
1.75 & 24 \\
2.15 & 30 \\
2.65 & 36 \\
3.05 & 42 \\
 & \\
\bottomrule
\end{tabular}
\end{subtable}
\end{table}

Durch diese wird anschließend eine Ausgleichsgerade der Form

\begin{equation}
  \frac{D}{D^2 + L^2} = a_i B + b_i
\end{equation}

gelegt. Der entsprechende Plot ist in \autoref{fig:plot-magnetisch} zu sehen.

\begin{figure}
  \centering
  \begin{subfigure}{0.49\textwidth}
    \centering
    \includegraphics[width=0.98\textwidth]{build/plot_magnetisch_0.pdf}
    \caption{$U_\text{B} = 250$ V}
  \end{subfigure}
  \begin{subfigure}{0.49\textwidth}
    \centering
    \includegraphics[width=0.98\textwidth]{build/plot_magnetisch_1.pdf}
    \caption{$U_\text{B} = 400$ V}
  \end{subfigure}
  \caption{Plots der Ablenkung $D$ gegen das ablenkende Magnetfeld $B$ bei verschiedenen Beschleunigungspannungen $U_\text{B}$.}
  \label{fig:plot-magnetisch}
\end{figure}

Mittels Python 3.7.0 werden die Koeffizienten als

\begin{align*}
  a_1 &= (8,6 \pm 0,1) \cdot 10^{3} \, \frac{\symup{m}}{\symup{T}} \\
  b_1 &= (3 \pm 1) \cdot 10^{-2} \, \symup{m} \\
  a_2 &= (7,70 \pm 0,08) \cdot 10^{3} \, \frac{\symup{m}}{\symup{T}} \\
  b_2 &= (6 \pm 9) \cdot 10^{-3} \, \symup{m}
\end{align*}

bestimmt.
Damit ist nach GLEICHUNG die spezifische Elektronenladung als 

\begin{align*}
  \bigg( \frac{e_0}{m_0} \bigg)_{250 \symup{V}} &= (1,48 \pm 0,03) \cdot 10^{11} \, \frac{\symup{C}}{\symup{kg}}
  \bigg( \frac{e_0}{m_0} \bigg)_{400 \symup{V}} &= (1,90 \pm 0,04) \cdot 10^{11} \, \frac{\symup{C}}{\symup{kg}}
\end{align*}

gegeben.