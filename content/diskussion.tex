\section{Diskussion}
\label{sec:Diskussion}

\subsection{Ablenkung im E-Feld}

Die experimentell bestimmte Größe für das Verhältnis zwischen $\frac{D}{U_\text{d}}$ und $\frac{1}{U_\text{B}}$ ist 

\begin{center}
    $\frac{p L }{2 d} = (330 \pm 10)$ mm.
\end{center}

Der Theoriewert beträgt $357,5 \, \symup{mm}$, was lediglich um $8,3 \%$ größer ist als der experimentell bestimmte Wert.
Auch die linearen Ausgleichsfits nähern die Messwert sehr gut an.

Zu bemerken ist, dass der leuchtende Punkt auf dem Schirm nicht sehr scharf ist.
Des Weiteren ist die Skala ohne Längenangaben vorliegend, sodass die einzelnen Abstände zusätzlich analog gemessen werden müssen.
So können die Abweichungen gut erklärt werden.

Der Wert der Frequenz der Sinusspannung beträgt $\nu_\text{Sinus} = (50,029 \pm 0,005) \, \symup{Hz}$.
Die Messwerte weisen eine gute lineare Abhängigkeit auf, was die geringe Standardabweichung von $0,001 \%$ erklärt.
Es ist jedoch zu bemerken, dass das Bild auf dem Schirm nie exakt starr einzustellen ist. Wenn eine gewisse Zeit gewartet wurde,
so begann das Bild sich erneut zu bewegen.



\subsection{Ablenkung im B-Feld}

Die experimentell bestimmen spezifischen Elektronenladungen betragen

\begin{align*}
  \bigg( \frac{e_0}{m_0} \bigg)_{250 \symup{V}} &= (1,48 \pm 0,03) \cdot 10^{11} \, \frac{\symup{C}}{\symup{kg}} \\
  \bigg( \frac{e_0}{m_0} \bigg)_{400 \symup{V}} &= (1,90 \pm 0,04) \cdot 10^{11} \, \frac{\symup{C}}{\symup{kg}}
\end{align*}

und der Literaturwert

\begin{center}
    $\bigg( \frac{e_0}{m_0} \bigg)_\text{Literatur} \approx 1,758 \cdot 10^{11} \, \frac{\symup{C}}{\symup{kg}}$ \cite{elektron}.
\end{center}

Für eine Beschleunigungsspannung von $U_B = 250 \, \symup{V}$ beträgt die Abweichung $16 \, \%$, für $400 \, \symup{V}$ beträgt sie $8 \, \%$.
Auch die linearen Fits nähern die Messwerte gut an-
Dieselben Ungenauigkeiten wie im ersten Versuchsteil treffen auch hier zu, womit die geringen Abweichungen erklärt werden können.

Im letzten Versuchsteil wird das Erdmagnetfeld als $B_\text{Erde} = 5,34 \, \symup{\mu T}$ bestimmt.
Der Literaturwert beträgt $B_\text{Erde, Lit} = 4,910 \, \symup{\mu T}$ \cite{erdmagnet}.
Es fällt auf, dass dieser fast 10 mal größer als der experimentell bestimmte Wert ist.
Dies kann jedoch dadurch erklärt werden, dass das Deklinatorium-Inklinatorium sehr veraltet ist und sich kaum bis gar nicht ausrichtet,
sodass die Richtung des Erdmagnetfeldes nur mit sehr starker Abweichung bestimmt werden kann.